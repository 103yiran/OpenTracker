%\documentclass[12pt,draft]{article}
\documentclass[12pt]{article}
\AtBeginDvi{\input{zhwinfonts}}
\usepackage{mathrsfs}
\usepackage{amsmath,amsthm,amsfonts,amssymb}
\usepackage{geometry}
\usepackage{fancyhdr}
\usepackage{indentfirst}
\usepackage{float}
\usepackage{listings}
\usepackage[dvips]{graphicx}
\usepackage{subfigure}
\usepackage[font=small]{caption}
\usepackage{threeparttable}
\usepackage{cases}
\usepackage{multicol}
\usepackage{url}
\usepackage{amsmath}
\usepackage{commath}
\usepackage{bm}
\usepackage{xcolor}
\usepackage{overpic}
\usepackage[utf8]{inputenc}
\usepackage[american]{babel}
%\usepackage[round]{natbib}
\usepackage{graphicx}
\numberwithin{equation}{section}
\geometry{left=1.5cm,right=1.5cm,top=1.5cm,bottom=1.5cm}
\setlength{\parskip}{0.3\baselineskip}
\setlength{\headheight}{15pt}
\usepackage{color}   %May be necessary if you want to color links
\usepackage{hyperref}
\hypersetup{
    colorlinks=true, %set true if you want colored links
    linktoc=all,     %set to all if you want both sections and subsections linked
    linkcolor=blue,  %choose some color if you want links to stand out
}
\setlength{\parindent}{4ex}
\begin{document}\small
\title{OpenTracker Implement Notes}
\author{rockking.jy@gmail.com}
\pagestyle{fancy}\fancyhf{}
\lhead{}\rhead{rockking.jy@gmail.com}
\lfoot{\textit{}}\cfoot{}\rfoot{\thepage}
\renewcommand{\headrulewidth}{1.pt}
\renewcommand{\footrulewidth}{1.pt}
\maketitle
\tableofcontents
\newpage
%=======================================
\section{Math Foundation}
We use \bm{$n$} for original Discrete Space Domain, \bm{$t$} or \bm{$x$} for original Continuos Space Domain. \par
\bm{$k$} for Discrete Fourier Domain, and \bm{$\xi$} for Continuos Fourier Domain. 
\subsection{Basic Notations}
Complex-value functions $g, h: \mathbb{R} \rightarrow \mathbb{C}$ are periodic with period $T > 0$. \par
Space $L^2(T)$ : Hilbert space equipped with an inner product $\langle \cdot , \cdot \rangle$ of periodic functions with period $T>0$. \par
For $g, h \in L^2(T)$, 
\begin{equation}\label{eq:conj}
	\langle g,h \rangle = \frac{1}{T} \int^{T}_{0} g(t) \overline{h(t)} dt
\end{equation}
where bar means complex conjugation. \par
Complex exponential functions: 
\begin{equation}
	e_k(t) = e^{i 2\pi kt/T}
\end{equation} \par
$\{e_k(t)\}^{\infty}_{-\infty}$ forms an orthonormal basis for $L^2(T)$. \par

%--------------------------------------------------
\subsection{Matrix Derivatives \cite{nasrabadi2007pattern}}
\begin{equation} \label{eq:matrixderivativ1}
	\frac{\partial}{\partial \bm{x}}(\bm{x^Ha}) = 
	\frac{\partial}{\partial \bm{x}}(\bm{a^Hx})= \bm{a}
\end{equation}

\begin{equation} \label{eq:matrixderivativ2}
	\frac{\partial}{\partial \bm{x}}(\bm{AB}) = 
	\frac{\partial \bm{A}}{\partial \bm{x}} \bm{B} + \bm{A} \frac{\partial \bm{B}}{\partial \bm{x}} 
\end{equation}

\begin{equation}\label{eq:matrixderivativ3}
	\frac{\partial}{\partial \bm{x}}(\bm{A^{-1}}) = 
	\bm{A}^{-1} \frac{\partial \bm{A}}{\partial \bm{x}} \bm{A}^{-1}
\end{equation}

\begin{equation}\label{eq:matrixderivativ4}
	\frac{\partial}{\partial \bm{x}}(\norm{\bm{Ax}}^2) 
	= \frac{\partial}{\partial \bm{x}} (\bm{x^HA^HAx} )
	= \bm{A^HAx + A^HAx} = 2\bm{A^HAx}
\end{equation}
%--------------------------------------------------
\subsection{Kronecker product} \label{ch:kronecker}
If \textbf{A} is an $m \times n$ matrix and \textbf{B} is a $p \times q$ matrix, then the Kronecker product $A \otimes B$ is the $mp  \times nq$ block matrix:
\begin{equation}
	\mathbf{A}\otimes\mathbf{B} = 
	\begin{bmatrix} 
		a_{11} \mathbf{B} & \cdots & a_{1n}\mathbf{B} \\ 
		\vdots & \ddots & \vdots 
		\\ a_{m1} \mathbf{B} & \cdots & a_{mn} \mathbf{B} 
	\end{bmatrix}_{mp\times nq}
\end{equation} \par
%--------------------------------------------------
\subsection{Forbenius norm} \label{ch:forbenius}
\begin{equation}
	\|A\|_{\rm F} = \sqrt{\sum_{i=1}^m \sum_{j=1}^n |a_{ij}|^2} = \sqrt{\operatorname{trace}(A^\dagger A)} = \sqrt{\sum_{i=1}^{\min\{m, n\}} \sigma_i^2(A)}
\end{equation}
%--------------------------------------------------
\subsection{Fourier transformation}
Fourier transformation:
\begin{equation}\label{eq:fourierTrans}
	\hat{f}(\xi) = \int_{-\infty}^{\infty} f(x)\ e^{-2\pi i x \xi}\,dx
\end{equation} \par
Inverse Fourier transformation: 
\begin{equation}
	f(x) = \int_{-\infty}^{\infty} \hat f(\xi)\ e^{2 \pi i x \xi}\,d\xi
\end{equation} \par

For \textbf{discrete} points: $\left \{ \mathbf{ x_n } \right \} := x_0, x_1, \ldots, x_{N-1}$ and 
$\left \{ \mathbf{X_k} \right \} := X_0, X_1, \ldots, X_{N-1}$, we have: \par
Discrete Fourier transformation: 
\begin{equation}\label{eq:fourierdisc}
	X_k = \sum_{n=0}^{N-1} x_n \cdot e^{-i2\pi kn/N}
\end{equation} \par
Inverse discrete Fourier transformation
\begin{equation}
x_n = \frac{1}{N}\sum_{k=0}^{N-1}X_k \cdot e^{i 2 \pi kn/N} 
\end{equation} \par 

Fourier transformation of $g \in L^2(T)$: 
\begin{equation}\label{eq:fourierL2T}
	\hat{g}[k] = \langle g , e_k \rangle = \frac{1}{T} \int^{T}_{0} g(t)e^{-i 2\pi kt/T} dt
\end{equation} \par

Any $g \in L^2(T)$ can be expressed as: $g(t)=\sum^{\infty}_{-\infty}\hat{g}[k] e_k$. \par
\textbf{Shift property in $L^2(T)$:}  \par
For any real number $L$, if $h(t)=g(t-L)$, then
\begin{equation} \label{eq:fouriershift}
	\hat{h}[k]=e^{-2\pi i k L / T} \hat{g}[k]
\end{equation}

\textbf{Parseval's formula:}
\begin{equation} \label{eq:parseval}
	\norm{g}^2_{L^2}=\norm{\hat{g}}^2_{l^2}
\end{equation}
where $\norm{g}^2= \langle g , g \rangle$ and $\norm{\hat{g}}^2_{l^2}=\sum^{\infty}_{-\infty}\abs{\hat{g}[k]}^2$. \par

\textbf{Poisson summation formula:}
\begin{equation}
	\sum_{n=-\infty}^\infty f(n)=\sum_{k=-\infty}^\infty \hat f\left(k\right)
\end{equation} \par
\begin{equation} \label{eq:poisson}
	\sum_{n=-\infty}^{\infty} s(t + nT)=\sum_{k=-\infty}^{\infty} \frac{1}{T}\cdot \hat s\left(\frac{k}{T}\right)e^{i 2\pi \frac{k}{T} t }
\end{equation} \par

\textbf{Symmetry of discrete Fourier transformation} \par
From equation (\ref{eq:fourierdisc}), we can have:
\begin{equation}
	X_{N-k} = \sum_{n=0}^{N-1} x_n \cdot e^{-i2\pi (N-k)n/N} = \sum_{n=0}^{N-1} x_n \cdot e^{-i2\pi n} e^{+i2\pi kn/N} 
	= \sum_{n=0}^{N-1}  x_n \cdot e^{+i2\pi kn/N} = \overline{X}_k
\end{equation} \par
So the Fourier coefficient is symmetry to the axis $(N+1)/2$, with a complex conjugation, this property could be used to save the memory and computing resources. \par
%--------------------------------------------------
\subsection{FFT}
FFT is short Fast Fourier Transformation, which fully use the symmetry property of Fourier transformation. \par
%--------------------------------------------------
\subsection{Correlation}
Correlation definition:
\begin{align} \label{eq:conv}
	(f \star g )(t) &\stackrel{\mathrm{def}}{=}\ \int_{-\infty}^\infty f^*(\tau) g(t + \tau) \, d\tau 
\end{align} \par
Discrete correlation:
\begin{align}
	(f \star g)[k] &= \sum_{l=-\infty}^\infty f^*[l] g[k + l] 
\end{align} \par
Correlation properties:
\begin{equation} \label{eq:correlationproperty1}
	\widehat{f \star g}=\hat{f}^* \hat{g}=\hat{f} \hat{g}^*
\end{equation}
where $^*$ means complex conjugate. \par
%--------------------------------------------------
\subsection{Convolution}
Convolution definition:
\begin{align} \label{eq:conv}
	(f * g )(t) &\stackrel{\mathrm{def}}{=}\ \int_{-\infty}^\infty f(\tau) g(t - \tau) \, d\tau \\
	&= \int_{-\infty}^\infty f(t-\tau) g(\tau)\, d\tau
\end{align} \par

Discrete convolution: 
\begin{align}
	(f * g)[k] &= \sum_{l=-\infty}^\infty f[l] g[k - l] \\
			&= \sum_{l=-\infty}^\infty f[k-l] g[l]
\end{align} \par

Circular convolution operation(with normalization) $*:L^2(T) \times L^2(T) \rightarrow L^2(T)$ here is defined by:
\begin{equation} \label{eq:circonv}
	 (g * h)(t) = \frac{1}{T} \int^{T}_{0} g(t-s) h(s) ds
\end{equation} \par
Convolution properties: 
\begin{equation}\label{eq:convprop}
	 \widehat{g*h}=\hat{g}\hat{h},\ \ \ \ \  \widehat{gh}=\hat{g}*\hat{h}
\end{equation} \par
To calculate discrete convolution, one of the inputs is converted into a \textbf{Toeplitz matrix}:
\begin{equation} \label{eq:convToep}
 y = h \ast x =
            \begin{bmatrix}
                h_1 & 0 & \ldots & 0 & 0 \\
                h_2 & h_1 & \ldots & \vdots & \vdots \\
                h_3 & h_2 & \ldots & 0 & 0 \\
                \vdots & h_3 & \ldots & h_1 & 0 \\
                h_{m-1} & \vdots & \ldots & h_2 & h_1 \\
                h_m & h_{m-1} & \vdots & \vdots & h_2 \\
                0 & h_m & \ldots & h_{m-2} & \vdots \\
                0 & 0 & \ldots & h_{m-1} & h_{m-2} \\
                \vdots & \vdots & \vdots & h_m & h_{m-1} \\
                0 & 0 & 0 & \ldots & h_m
            \end{bmatrix}_{(m+n-1) \times n}
            \begin{bmatrix}
                x_1 \\
                x_2 \\
                x_3 \\
                \vdots \\
                x_n
            \end{bmatrix}
\end{equation}
or 
\begin{equation}
            \begin{bmatrix}
                h_1 & h_2 & h_3 & \ldots & h_{m-1} & h_m
            \end{bmatrix}
            \begin{bmatrix}
                x_1 & x_2 & x_3 & \ldots & x_n & 0 & 0 & 0& \ldots & 0 \\
                0 & x_1 & x_2 & x_3 & \ldots & x_n & 0 & 0 & \ldots & 0 \\
                0 & 0 & x_1 & x_2 & x_3 & \ldots & x_n & 0  & \ldots & 0 \\
                \vdots & \vdots & \vdots & \vdots & \vdots & \ldots & \vdots & \vdots  & \ldots & 0 \\
                0 & \ldots & 0 & 0 & x_1 & \ldots & x_{n-2} & x_{n-1} & x_n & \vdots \\
                0 & \ldots & 0 & 0 & 0 & x_1 & \ldots & x_{n-2} & x_{n-1} & x_n
            \end{bmatrix}_{m \times (m+n-1)}
\end{equation}
%--------------------------------------------------
\subsection{Newton Method} \label{ch:newtonmethod}

%--------------------------------------------------
\subsection{Gauss-Newton Method} \label{ch:gaussnewton}

%--------------------------------------------------
\subsection{Conjugate Gradient \cite{shewchuk1994introduction}} \label{ch:conjugategradient}
%==========================================
\section{HOG feature \cite{dalal2005histograms} \cite{felzenszwalb2010object}} 
\subsection{Original HOG feature  \cite{dalal2005histograms} } \label{ch:hog}
Basic idea: local object appearance and shape can often be characterized rather well
by the distribution of local intensity gradient or edge directions, even without precise knowledge
of the corresponding gradient or edge position. \par
Input - detection window with size: $64 \times 128$. 
\begin{enumerate}
	\item Divide the detection window into $16 \times 16$ blocks of $50\%$ overlap.
		\begin{itemize}
			\item $7 \times 15 = 105$ blocks in total ($(64 \div (16 \div 2) - 1) \times 
				(128 \div (16 \div 2) - 1)$. 
		\end{itemize}
	\item Each block consists of $2 \times 2$ cells with size 8 pixels $ \times $ 8 pixels.
	\item Compute gradients for each pixel.
		\begin{itemize}
			\item Using $[-1, 0, 1]$ gradient filter.
		\end{itemize}
	\item Quantize the gradient orientation into 9 orientations bins in $0^\circ-180^\circ$
	(with $20^\circ$ interval).
		\begin{itemize}
			\item The vote is the gradient magnitude. \\
			For color images, calculate separate gradients for each color channel, and take the one with 
			the largest norm at the pixel's gradient vector.
			\item Interpolate votes bi-linearly between neighboring bin center. \\
			 For example, suppose a pixel have orientation $80^\circ$, next to it has $70^\circ$ and 
			 $90^\circ$ bins, then: $85-70=15$, $90-85=5$, so, for bin $70^\circ$ it contributes:
			  $15/(15+5)=3/4 \times$ gradient magnitude, and for bin $90^\circ$: $5/(15+5)=1/4 \times$
			  gradient magnitude.
			\item The vote can also be weighted by Gaussian to down-weight near the edge.
		\end{itemize}
	\item For each cell accumulating the 1-D histogram of gradient directions or edge orientations over
	the pixels of the cell.
	\item Normalization for blocks, for better invariance to illumination, shadowing etc..
		\begin{itemize}
			\item L2-Hys (Lowe-style clipped L2 norm) block normalization. That is, L2-norm followed by 
			clipping (limiting the maximum values of v to 0.2) and renormalizing.
		\end{itemize}		
	\item Concatenate histograms.
		\begin{itemize}
			\item Feature dimension: $36(=4 \times 9)$ for each block (in this example, 105 blocks in total,
			so $36 \times 105$ features in total for this input detection window).
		\end{itemize}	
\end{enumerate}
%--------------------------------
\subsection{Improved HOG feature \cite{felzenszwalb2010object}}
Basic idea: using 13-dimensional feature instead of previous 36-dimensional with no significant effect, and
add  contrast sensitive and contrast insensitive features to a 31-dimensional feature vector to improve the 
performance. \par
Input - image size: $w \times h$.
\begin{enumerate}
	\item Pixel-Level Feature Maps. 
		\begin{itemize}
			\item For each pixel (x, y), let $\theta(x,y)$ be the orientation and $r(x,y)$ be the magnitude of 
			the intensity gradient using $[-1, 0, 1]$ as gradient filter.
			\item The gradient calculated at the pixel is discretized into one of p values by contrast sensitive
			($B_1$) or contrast insensitive ($B_2$):
				\begin{equation}
					B_1(x,y) = round(\frac{p \theta(x,y)}{2 \pi})\mod p 
				\end{equation}
				\begin{equation}
					B_2(x,y) = round(\frac{p \theta(x,y)}{\pi})\mod p
				\end{equation}
			\item The feature vector $F(x, y)_b$ at pixel (x, y) is: 
				\begin{equation}
					F(x,y)_b=
					\begin{cases}
						r(x,y) & \text{if } b = B(x,y) \\
						0 & \text{otherwise}
    					\end{cases}
				\end{equation}
			where $b \in \{0, \cdots , p-1 \}$.
			\item Now we have a pixel-level feature map for the image.
		\end{itemize}
	\item Spatial Aggregation.
		\begin{itemize}
			\item Let $k > 0$ be the cell size.
			\item Aggregate pixel-level feature vectors to obtain a cell-based feature map C(i,j).
			\item In C(i,j), where $0 \le i \le \lfloor{(w-1)/k}\rfloor$, $0 \le j \le \lfloor{(h-1)/k}\rfloor$.
			\item Pixel (x, y) maps to $(\lfloor{x/k}\rfloor, \lfloor{y/k}\rfloor)$.
			\item The feature vector at a cell is the sum (or average) of the pixel-level features in that cell.
			\item Each pixel contributes to the feature vectors in the four cells around it using 
			bilinear interpolation.
			\item This aggregation provides some invariance to small deformations and reduces the size
			of a feature map.
		\end{itemize}
	\item Normalization and Truncation.
		\begin{itemize}
			\item Normalization provides the invariance to gain, while gradients provides invariant to bias.
			\item Let $T_{\alpha}(v)$ denote the truncation of a vector  v by $\alpha$.
			\item The feature map is obtained by concatenating the result of normalizing the map C with respect
			 to each normalization factor followed by truncation:
			 	\begin{equation}
					H(i,j)=
					\begin{bmatrix}
						T_{\alpha}(C(i,j)/N_{-1, -1}(i,j)) \\
						T_{\alpha}(C(i,j)/N_{+1, -1}(i,j)) \\
						T_{\alpha}(C(i,j)/N_{+1, +1}(i,j)) \\
						T_{\alpha}(C(i,j)/N_{-1, +1}(i,j)) \\
					\end{bmatrix}
				\end{equation}
			\item We use four normalization factors for C(i,j): $N_{\delta, \gamma}(i,j)$ with 
			$\delta, \gamma \in \{-1, 1\}$,
				\begin{equation}
					N_{\delta, \gamma}(i,j)=(\norm{C(i,j)}^2+\norm{C(i+\delta,j)}^2+
									\norm{C(i,j+\gamma)}^2+\norm{C(i+\delta,j+\gamma)}^2)^{1/2}
				\end{equation}
			Each factor measures the "gradient energy" in a square block of four cells containing (i,j).
			\item Commonly, HOG features using $p=9$, discretized with $B_2$, cell size $k=8$ and truncation 
			$\alpha = 0.2$, so we have 36-dimensional feature vector (9 orientations $ \times 4$ 
			normalizaions).
		\end{itemize}		
	\item PCA and Analytic Dimensionality Reduction.
		\begin{itemize}
			\item 9 contrast insensitive orientations, 18 contrast sensitive orientations, 4 normalization factors, 
			so we have $4 \times (9+18)=108$ dimensional feature vectors. 
			\item Use an analytic projection of 108 dimensional features to:
				\begin{enumerate}
			 		\item 27 (=9+18) sums over 4 different normalizations.
					\item 4 dimensional features sums over 9 contrast insensitive orientations. 
				\end{enumerate}
			\item In total, the final feature map has 31-dimensional vectors (27+4).
		\end{itemize}	
\end{enumerate}
%=======================================
\section{CN feature \cite{van2007learning}}
Color names (CN) are linguistic color labels assigned by humans to represent colors in the world. In English, it contains eleven basic color terms: black, blue, brown, green, orange, pink, purple, red, white and yellow. \par
Color naming is an operation that associates RGB observations with linguistic color labels.  \par
In \cite{van2007learning}, it provides the mapping automatically learned from images retrieved with Google-image search. It maps the RGB values to a probabilistic 11 dimensional color representation which sums up to 1. \par
%=======================================
\section{Deep features \cite{chatfield2014return}}
	\begin{table}[h!] 
  		\centering
  		\begin{tabular}{c|c|c|c|c|c|c|c}
   			Layer & Name & size & stride&rFieldStride& pad        & input         & output\\
   			\hline
			0        & Input   &[224,224,3,10]&           &1&              & 224 x 224 & 224 x 224\\ 
			1        & conv1 & [7,7,3,96]       &[2, 2]   &1& [0,0,0,0] & 224 x 224 & 109 x 109\\ 
			2        & relu1   &                       &            &2&               & 109 x 109 & 109 x 109\\ 
	     \textbf{3}  & \textbf{norm1} &                       &           &2&                & 109 x 109 & 109 x 109\\ 
			4        & pool1  & [3, 3]              & [2, 2]   &2& [0,1,0,1]  & 109 x 109  & 54 x 54\\ 
			5        & conv2 & [5,5,96,256]   & [2, 2]   &4&[1,1,1,1]   &  54 x 54    & 26 x 26 \\ 
			6        & relu2   &                       &            &8&                &  26 x 26 &  26 x 26\\ 
			7        & norm2 &                       &            &8&                &  26 x 26 &  26 x 26\\ 
			8        & pool2  & [3, 3]              & [2, 2]   &8& [0,1,0,1]  &  26 x 26 & 13 x 13\\ 
			9        & conv3 & [3,3,256,512] & [1, 1]   &16&[1,1,1,1]   & 13 x 13 & 13 x 13\\ 
			10      & relu3   &                       &            &16&                & 13 x 13 & 13 x 13\\
			11      & conv4 & [3,3,512,512]  & [1, 1]  &16&[1,1,1,1]   & 13 x 13 & 13 x 13\\ 
			12      & relu4   &             		&          &16&                 & 13 x 13 & 13 x 13\\ 
			13      & conv5 & [3,3,512,512]  & [1, 1] &16&[1,1,1,1]    & 13 x 13 & 13 x 13\\ 
	     \textbf{14} & \textbf{relu5}   &                        &          &16&                 & 13 x 13 & 13 x 13\\  
  		\end{tabular}
		\caption{Neural Network Structure used in C-COT(Chapter \ref{ch:ccot}) and ECO(Chapter \ref{ch:eco})}
		\label{tab:deep}
	\end{table} \par
Some trackers like C-COT, ECO etc. using deep features\cite{Goodfellow-et-al-2016-Book} to get the abstract high-level feature of the image. \par
Equation to calculate output size from input for convolution layer:
\begin{equation}
	\abs{Ouput} = \lfloor{\frac{\abs{Input} - size + pad1 + pad2}{stride}}\rfloor + 1
\end{equation}	 \par
Table (\ref{tab:deep}) is the network structure used in C-COT(Chapter \ref{ch:ccot}) and ECO(Chapter \ref{ch:eco}), and it just use the \textbf{3th} and \textbf{14th} layer. \par
%=======================================
\section{MOSSE \cite{bolme2010visual}}
$x_i$: training images ; $f$: filter; $g_i$: actual target; $y_i$: desired target, here we choose 2D Gaussian centered on the target in training image. \par
The correlation filter performs on the training image, gave the target that we want to tracking:
\begin{equation}
	x_i \star f = g_i
\end{equation}\par
The maximum point of $g_i$ is the centre of the target. \par
With the property of correlation (\ref{eq:correlationproperty1}), we have:
\begin{equation}
	\widehat{x_i \star f} = \hat{x}_i \odot \hat{f}^*
\end{equation}
$\odot$ means element-wise multiplication. And this could speed up the localization procedure.\par
%------------------------
\subsection{Localization}
\begin{enumerate}
	\item Do Fourier transform of the input image: $\hat{x}_i=\mathcal{F}(x_i)$, and of the filter: $\hat{f}=\mathcal{F}(f)$. \par
	\item Do the element-wise multiplication: $\hat{g}_i =\hat{x}_i \odot \hat{f}^*$. \par
	\item Transform back to the spatial domain using the inverse FFT to $g_i=\mathcal{F}^{-1}(\hat{g}_i)$. \par
	\item The maximum point is the location of the target.
\end{enumerate}
%------------------------
\subsection{Training}
To train a filter, MOSSE minimizes the error between actual output and the desired output:
\begin{equation} \label{eq:mossetarget}
	\min_{\hat{f}} \sum_{i=1}^{m} |\hat{x}_i \odot \hat{f}^* - \hat{y}_i|^2
\end{equation} \par
Because correlation in the Fourier domain is an element- wise multiplication, each element of the filter can be optimized independently.
\begin{equation}
	\min_{\hat{f}_{\omega \nu}}\sum_{i=1}^{m}|\hat{x}_{i\omega \nu}  \hat{f}^*_{\omega \nu} - \hat{y}_{i\omega \nu}|^2
\end{equation} 
For each element of $\hat{f}$, we do the partial, and equals to zero to get the minimum point:
\begin{equation}
	0= \frac{\partial}{\partial f^*_{\omega \nu}}\sum_{i=1}^{m}|\hat{x}_{i\omega \nu} \odot \hat{f}^*_{\omega \nu} - \hat{y}_{i\omega \nu}|^2
\end{equation} 
\begin{equation}
	\hat{f}= \frac{\sum_{i=1}^{m}\hat{x}_i \odot \hat{y}^*_i }{\sum_{i=1}^{m}\hat{x}_i \odot \hat{x}^*_i}
\end{equation} 
%------------------------
\subsection{Update method}
For each new sample $i$:
\begin{align}
	\hat{f}_i &= \frac{A_i}{B_i} \\
	A_i &= \eta \hat{x}_i \odot \hat{y}_i^* + (1-\eta)A_{i-1} \\
	B_i &= \eta\hat{x}_i\odot \hat{x}_i^* + (1-\eta)B_{i-1}
\end{align}
where $\eta$ is the learning rate.

%------------------------
\subsection{Failure Detection}
Peak to Sidelobe Ratio (PSR): g is split into the peak which is the maximum value and the sidelobe which
is the rest of the pixels excluding an $11 \times 11$ window around the peak. The PSR is then defined as:
\begin{equation}
	PSR = \frac{g_{max}-\mu_{s1}}{\theta_{s1}}
\end{equation}
where $g_{max}$ is the peak values and $\mu_{s1}$ and $\theta_{s1}$ are the mean and standard deviation of the sidelobe. \par
%=======================================
\section{CSK \cite{henriques2012exploiting}}

%=======================================
\section{KCF / DCF \cite{henriques2015high}}

%=======================================
\section{C-COT \cite{DanelljanECCV2016}} \label{ch:ccot}
%--------------------------------------------------
\subsection{Continuous Learning Formulation}
$x_j$: Training Sample $j \in [1, ..., J]$. \par
$d \in \{1, \cdots , D\}$: Feature channel of training samples (ex. if choose deep layer 1, 5 as features, then $D = 2$). \par
$N_d$: Number of spatial sample points in feature channel d (ex. for HOG feature $N_{HOG}=31 \times 105$ (chapter \ref{ch:hog})). \par
$\chi=\mathbb{R}^{N_1} \times \cdots \times \mathbb{R}^{N_D}$. \par
$\{x^d_j[n]\}, n \in \{0, \cdots, N_d -1\}$: Feature points in channel d of training sample $x_j$.\par
$[0, T) \subset \mathbb{R} $: Spatial support of the feature map, where $T$ is the size of the support region.\par
$J_d$: Interpolation operator of training sample x in feature channel d, defined as: 
\begin{equation}
	J_d\{x^d\}(t)=\sum^{N_d-1}_{n=0} x^d[n] b_d(t-\frac{T}{N_d} n)
\end{equation} \par
$f^d \in L^2(T)$ is the continuous filter for feature channel d. \par
$S_f: \chi \rightarrow L^2(T)$: maps a sample $x \in \chi$ to a confidence score function defined on the interval $[0, T)$, defined as: 
\begin{equation} \label{eq:score}
	S_f\{x\}=\sum^D_{d=1} f^d * J_d\{x^d\}, \  x \in \chi.
\end{equation} 
the convolution here is the circular convolution in continuous domain as defined in (\ref{eq:conv}). \par
The target is localized by maximizing the confidence score in an image region. \par

The filter $f$ is trained by minimizing the functional: 
\begin{equation} \label{eq:target}
	E(f)=\sum^{m}_{j=1} \alpha_j \norm{S_f\{x_j\}-y_j}^2_{L^2} + \sum^D_{d=1} \norm{w f^d}^2_{L^2}
\end{equation}

%--------------------------------------------------
\subsection{Training the filter $f$ in Fourier domain}
To train the filter f, we minimize the function (\ref{eq:target}) in the Fourier domain. \par
By (\ref{eq:fouriershift}) and (\ref{eq:fourierdisc}), we have:
\begin{align}\begin{split}
	\widehat{J_d\{x^d\}}[k]&=\sum^{N_d-1}_{n=0} x^d[n] \widehat{b_d(t-\frac{T}{N_d} n)}[k] \\
	&=\sum^{N_d-1}_{n=0} x^d[n] e^{-i\frac{2\pi}{N_d}kn} \hat{b}_d[k] \\
	&=\hat{b}_d[k] \sum^{N_d-1}_{n=0} x^d[n] e^{-i\frac{2\pi}{N_d}kn}  \\
	&=X^d[k] \hat{b}_d[k] 
\end{split}\end{align} \par 
With the property of (\ref{eq:convprop}), we have:
\begin{equation} \label{eq:scoreFourier}
	\widehat{S_f\{x\}}[k]= \sum^D_{d=1} \hat{f}^d[k] X^d[k] \hat{b}_d[k] 
\end{equation} \par
By using Parseval's formula (\ref{eq:parseval}):
\begin{equation} \label{eq:targetFourier}
	E(f)=\sum^{m}_{j=1} \alpha_j \norm{\sum^D_{d=1}\hat{f}^dX^b_j\hat{b}_d-\hat{y}_j}^2_{l^2}
		 + \sum^D_{d=1} \norm{\hat{w} * \hat{f}^d}^2_{l^2}
\end{equation}
The functional $E(f)$ can be minimized with respect of $\hat{f}^d[k]$. \par 

%------------------------------------------------
\subsection{Training the filter $\hat{\bm{f}}$ as finite matrix}
In practice the filter $f$ needs to be represented by a \textbf{finite} set. So we obtain a \textbf{finite} representation by minimizing (\ref{eq:targetFourier}) over the finite-dimensional subspace $V = span\{e_k\}^{K_1}_{-K_1} \times \cdots \times span\{e_k\}^{K_D}_{-K_D} \subset L^2(T)^D$, by assuming $\hat{f}^d[k] = 0$ for $\abs{k} > K_d$. Here we set $K_d = \lfloor \frac{N_d}{2} \rfloor$. \par
Define: 
\begin{equation}
	\hat{\bm{f}}^d= 
	\begin{bmatrix}
		\hat{f}^d[-K_d] \\  \vdots \\ \hat{f}^d[K_d] 
	\end{bmatrix}_{(2K_d+1) \times 1}
	 \in \mathbb{C}^{2K_d+1} \text{,   }
	\hat{\bm{f}}= 
	\begin{bmatrix}
		\hat{\bm{f}}^1 \\ \hat{\bm{f}}^2 \\ \vdots \\ \hat{\bm{f}}^D
	\end{bmatrix}_{\sum^{D}_{d=1}(2K_d+1) \times 1}
	 \text{,   }
	 \hat{\bm{y}}_j= 
	\begin{bmatrix}
		\hat{y}_j[-K] \\  \vdots \\ \hat{y}_j[K] 
	\end{bmatrix}_{(2K+1) \times 1}
\end{equation}
where $K:=\max_dK_d$.  \par
And define: 
\begin{equation}
	A^d_j = 
	\begin{bmatrix}
		X^d_j[-K_d]\hat{b}_d[-K_d] & \cdots & 0 \\
		\vdots & \ddots & \vdots \\
		0 & \cdots & X^d_j[K_d]\hat{b}_d[K_d] \\
	\end{bmatrix}_{(2K_d+1)\times(2K_d+1)}
\end{equation} \par
\begin{equation}
	A_j = 
	\begin{bmatrix}
		0_{(K-K_1)\times(2K_1+1)} & \cdots & 0_{(K-K_D)\times(2K_D+1)}  \\
		A_j^1 & \cdots & A_j^D \\
		0_{(K-K_1)\times(2K_1+1)} & \cdots & 0_{(K-K_D)\times(2K_D+1)} 
	\end{bmatrix}_{(2K+1)\times \sum^D_{d=1}(2K_d+1)}
\end{equation} \par
Let $L$ be the number of non-zero coefficients $\hat{w}[k]$, such that  $\hat{w}[k]=0$ for all $\abs{k}>L$. \par
Define $W_d$ to be the $(2K_d+2L+1) \times (2K_d+1)$ Toeplitz matrix corresponding to 
the convolution operator $W_d\hat{\bm{f}}^d=vec \ \hat{w}*\hat{f}^d$ (\ref{eq:convToep}):
\begin{equation} 
	W_d\hat{\bm{f}}^d =
	\begin{bmatrix}
                \hat{w}[-K_d]     & 0 				& \ldots & 0 			& 0 \\
                \hat{w}[-K_d+1] & \hat{w}[-K_d] 		& \ldots & \vdots 		& \vdots \\
                \hat{w}[-K_d+2] & \hat{w}[-K_d+1] 	& \ldots & 0 			& 0 \\
                \vdots 		&  \hat{w}[-K_d+2] 	& \ldots &  \hat{w}[-K_d] 	& 0 \\
                 \hat{w}[K_d-1]  & \vdots 			& \ldots &  \hat{w}[-K_d+1] &  \hat{w}[-K_d] \\
                 \hat{w}[K_d] 	&  \hat{w}[K_d-1] 	& \vdots & \vdots 		&  \hat{w}[-K_d+1] \\
                0 			&  \hat{w}[K_d]		& \ldots &  \hat{w}[K_d-2] 	& \vdots \\
                0 			& 0 				& \ldots &  \hat{w}[K_d-1]	&  \hat{w}[K_d-2] \\
                \vdots 		& \vdots 			& \vdots &  \hat{w}[K_d] 	&  \hat{w}[K_d-1]\\
                0 			& 0 				& 0 	     & \ldots 		& \hat{w}[K_d]
	\end{bmatrix}_{(2K_d+2L+1) \times (2K_d+1)}
	\begin{bmatrix}
		\hat{f}^d[-K_d] \\  \vdots \\ \hat{f}^d[K_d] 
	\end{bmatrix}_{(2K_d+1) \times 1}
\end{equation} \par
Define $W=W_1 \oplus \cdots \oplus W_D$. \par
Then the matrix version of (\ref{eq:targetFourier}) becomes:
\begin{equation} \label{eq:targetMatrix}
	E_V(f)=\sum^{m}_{j=1} \alpha_j \norm{A_j \hat{\mathbf{f}}-\hat{\mathbf{y}}_j}^2_{2}
		 + \norm{W  \hat{\mathbf{f}}}^2_{2}
\end{equation} \par

%--------------------------------------------------
\subsection{The normal equations of $\hat{\bm{f}}$}
Define the matrix with J samples: 
\begin{equation}
	A = 
	\begin{bmatrix}
		A_1 \\ A_2 \\ \vdots \\ A_J
	\end{bmatrix}_{(2K+1)J \times \sum^D_{d=1}(2K_d+1)}
	 \text{,   }
	\hat{\bm{y}}= 
	\begin{bmatrix}
		\hat{\bm{y}}_1 \\ \hat{\bm{y}}_2 \\ \vdots \\ \hat{\bm{y}}_J
	\end{bmatrix}_{(2K+1)J \times 1}
	 \text{,   }
	 \Gamma=\alpha_1 I  \oplus \cdots \oplus \alpha_J I
\end{equation} \par

The equation (\ref{eq:targetMatrix}) becomes:
\begin{align} \begin{split} \label{eq:targetDerivative}
	E_V(f)&=
		(A \hat{\mathbf{f}}-\hat{\mathbf{y}})^H \Gamma (A \hat{\mathbf{f}}-\hat{\mathbf{y}})
		 + \hat{f}^H W^HW \hat{f} \\
		 &= (\hat{\mathbf{f}}^H A^H - \hat{\mathbf{y}}^H)\Gamma (A \hat{\mathbf{f}}-\hat{\mathbf{y}})
		 + \hat{f}^H W^HW \hat{f} \\
		 &= \hat{\mathbf{f}}^H A^H \Gamma A \hat{\mathbf{f}} - \hat{\mathbf{f}}^H A^H \Gamma \hat{\mathbf{y}}
		 - \hat{\mathbf{y}}^H \Gamma A \hat{\mathbf{f}} + \hat{\mathbf{y}}^H \Gamma \hat{\mathbf{y}}
		 + \hat{f}^H W^HW \hat{f} 
\end{split}\end{align} \par
Do the derivative to $\hat{\bm {f}}$ for the equation(\ref{eq:targetDerivative}) by using the properties (\ref{eq:matrixderivativ1}) and (\ref{eq:matrixderivativ4}):
\begin{align}
	A^H \Gamma (A \hat{\bm{f}} - \hat{\bm{y}})  
	+ W^HW\hat{\bm{f}} = 0
\end{align} \par
So the minimizer of (\ref{eq:targetMatrix}) is found by solving the equations:
\begin{equation} \label{eq:ccot}
	(A^H \Gamma A + W^H W) \hat{\bm{f}} = A^H \Gamma \hat{\bm{y}}
\end{equation}
%--------------------------------------------------
\subsection{The desired output $y_j$} 
$g_T(t)$: T-periodic repetition of function g, defined as:
\begin{equation} \label{eq:periodic}
	g_T(t) = \sum^{\infty}_{n=-\infty} g(t-nT)
\end{equation} \par
$\hat{g}_T[k]$: Fourier transformation of $g_T$, defined as:
\begin{align}\begin{split}\label{eq:fouriergT}
	\hat{g}_T[k] 
	&\stackrel{(\ref{eq:fourierL2T})}{=}  \langle g_T , e_k \rangle \\
	&\stackrel{(\ref{eq:conj})}{=} \frac{1}{T} \int^{T}_{0} g_T(t) e^{-i 2 \pi k t / T} dt \\ 
	&\stackrel{(\ref{eq:periodic})}{=} \frac{1}{T} \int^{T}_{0} \sum^{\infty}_{n=-\infty} g(t-nT) e^{-i 2 \pi k t / T} dt \\
	&\stackrel{(\ref{eq:poisson})}{=} \frac{1}{T} \int^{T}_{0} \frac{1}{T} \sum^{\infty}_{k'=-\infty} \hat{g}(\frac{k'}{T}) e^{i 2 \pi k' t / T}  e^{-i 2 \pi k t / T} dt \\
	&\stackrel{}{=} \frac{1}{T}  \int^{T}_{0} \sum_{k' \neq k} \hat{g}(\frac{k'}{T}) e^{i 2 \pi (k'-k) t / T} d(\frac{t}{T}) +  \frac{1}{T} \hat{g}(\frac{k}{T})\\
	&= Const \times e^{i 2 \pi (k'-k) t / T} |^T_0+  \frac{1}{T} \hat{g}(\frac{k}{T})\\
	&= \frac{1}{T} \hat{g}(\frac{k}{T})
\end{split}\end{align} \par
$y_j(t)=\sum^{\infty}_{n=-\infty}z_j(t-nT)$ is the periodic repetition of the Gaussian function $z_j(t)=e^{-\frac{1}{2\sigma^2}(t-u_j)^2}$, here $u_j \in [0, T)$ is the estimated location of the target in the corresponding sample $x_j$. \par
The Fourier transformation of $z_j(t)$ is: 
\begin{align}\begin{split}\label{eq:fourierz}
	\hat{z}_j[k]
	&\stackrel{(\ref{eq:fourierTrans})}{=} \int^{\infty}_{-\infty} z_j(t) e^{-i 2 \pi k t} dt \\ 
	&\stackrel{}{=} \int^{\infty}_{-\infty} e^{-\frac{1}{2\sigma^2}(t-u_j)^2} e^{-i 2 \pi k t} dt \\ 
	&\stackrel{}{=} \int^{\infty}_{-\infty} e^{-\frac{1}{2\sigma^2}\tau^2} e^{-i 2 \pi k (\tau+u_j)} d\tau \\ 
	&\stackrel{}{=} e^{-i 2 \pi k u_j} \int^{\infty}_{-\infty} e^{-\frac{1}{2\sigma^2}\tau^2} e^{-i 2 \pi k \tau} d\tau \\ 
	&\stackrel{}{=} e^{-i 2 \pi k u_j} e^{-2(\sigma \pi k)^2} \int^{\infty}_{-\infty} e^{-\frac{1}{2\sigma^2}(\tau+ i 2 \pi \sigma^2 k)^2} d\tau \\ 
	&\stackrel{}{=} \sqrt{2\pi\sigma^2}e^{-i 2 \pi k u_j} e^{-2(\sigma \pi k)^2} 
\end{split}\end{align} \par
So the Fourier transformation of $y_j(t)$ is:
\begin{align}\begin{split}
	\hat{y}_j[k]
	&\stackrel{(\ref{eq:fouriergT})}{=} \frac{1}{T} \hat{z}_j(\frac{k}{T}) \\
	&\stackrel{(\ref{eq:fourierz})}{=} \frac{\sqrt{2\pi\sigma^2}}{T}e^{-i 2 \pi \frac{k}{T} u_j-2(\sigma \pi \frac{k}{T})^2} 
\end{split}\end{align} \par
\begin{align}\begin{split}
	\hat{y}_j=
	\begin{bmatrix}
		\hat{y}_j[-K] \\ \vdots \\ \hat{y}_j[K] 
	\end{bmatrix}_{(2K+1) \times 1}
\end{split}\end{align}
where $K=max_d K_d$.
%--------------------------------------------------
\subsection{Interpolation function $b_d$}
Standard cubic spline kernel \cite{szeliski2010computer} (3.79): 
\begin{equation}
	b(t)=
	\begin{cases}
		(a+2)\abs{t}^3-(a+3)t^2+1 & \text{if } \abs{t} < 1 \\
		a\abs{t}^3-5at^2+8a\abs{t}-4a & \text{if } 1 \leq \abs{t} < 2\\
		0	& \text{otherwise,}
    	\end{cases}
\end{equation} \par
First rescaling b to the sample interval $T/N_d$, then shifted half an interval $T/(2N_d)$ to align the origin of the continuous coordinate system with the sampling intervals of the feature map: 
\begin{equation}
	c_d(t)=b(\frac{N_d}{T}(t-\frac{T}{2N_d}))
\end{equation} \par
The Fourier transformation of it is:
\begin{align}\begin{split}\label{eq:fouriercd}
	\hat{c}_d(\xi) &= \int^{\infty}_{-\infty} c_d(t) e^{-i 2 \pi \xi t}dt \\
			&= \int^{\infty}_{-\infty} b(\frac{N_d}{T}(t-\frac{T}{2N_d})) e^{-i 2 \pi \xi t}dt \\
			&= \int^{\infty}_{-\infty} b(\frac{N_d}{T}\tau) e^{-i 2 \pi \xi (\tau + \frac{T}{2N_d} )} d\tau \text{, where } \tau = t-\frac{T}{2N_d}\\
			&= e^{-i\pi \frac{T}{N_d} k} \frac{T}{N_d} \int^{\infty}_{-\infty} b(\tau')e^{-i2\pi \xi \frac{T}{N_d} \tau' } d\tau' \text{, where } \tau' = \frac{N_d}{T}\tau\\
			&= \frac{T}{N_d} e^{-i\pi \frac{T}{N_d} k} \hat{b}(\frac{T}{N_d} \xi) 
\end{split}\end{align} \par
$b_d(t)=\sum^{\infty}_{n=-\infty}c_d(t-nT)$  is the periodic repetition of $c_d(t)$, its Fourier transformation is:
\begin{align}\begin{split}
	\hat{b}_d[k] 
	&\stackrel{(\ref{eq:fouriergT})}{=} \frac{1}{T} \hat{c}_d(\frac{k}{T}) \\
	&\stackrel{(\ref{eq:fouriercd})}{=}\frac{1}{N_d}e^{-i\frac{\pi}{N_d}k} \hat{b}(\frac{k}{N_d})
\end{split}\end{align} \par
where the Fourier transformation(\ref{eq:fourierTrans}) of $b(t)$ is:
\begin{align}\begin{split}
	\hat{b}[\xi] &=  \int_{-\infty}^{\infty} b(t)\ e^{-2\pi i t \xi} dt \\
	&=  \int_{-2}^{-1}\{a\abs{t}^3-5at^2+8a\abs{t}-4a \}e^{-2\pi i t \xi} dt  + \int_{-1}^{1}\{(a+2)\abs{t}^3-(a+3)t^2+1\} e^{-2\pi i t \xi}dt \\
	&\ \ \ \ + \int_{1}^{2}\{a\abs{t}^3-5at^2+8a\abs{t}-4a \}e^{-2\pi i t \xi} dt \\
	&=  \int_{-2}^{-1}\{-at^3-5at^2-8at-4a \}e^{-2\pi i t \xi} dt  + \int_{-1}^{0}\{-(a+2)t^3-(a+3)t^2+1\} e^{-2\pi i t \xi}dt \\
	&\ \ \ \ + \int_{1}^{2}\{at^3-5at^2+8at-4a \}e^{-2\pi i t \xi} dt   + \int_{0}^{1}\{(a+2)t^3-(a+3)t^2+1\} e^{-2\pi i t \xi}dt \\
	&= \int_{1}^{2}(at^3-5at^2+8at-4a )(e^{-2\pi i t \xi}-e^{2\pi i t \xi})dt  + \int_{0}^{1}\{(a+2)t^3-(a+3)t^2+1\} (e^{-2\pi i t \xi}-e^{2\pi i t \xi})dt\\
	&= \int_{1}^{2}(at^3-5at^2+8at-4a )(-2i)\sin(2\pi t \xi)dt  + \int_{0}^{1}\{(a+2)t^3-(a+3)t^2+1\}(-2i)\sin(2\pi t \xi)dt\\
	&= \frac{6(1-cos2\pi \xi)+3a(1-cos4\pi \xi) - (6+8a) \pi \xi \sin 2 \pi \xi - 2a \pi \xi \sin 4\pi \xi}{4\xi^4\pi^4}
\end{split}\end{align} \par
The last line just calculate with some attention. \par
Some calculation notes: 
\begin{align}\begin{split}
	\int_{b}^{c} \sin(2\pi t \xi)dt = \frac{1}{2\pi \xi} [\cos(2\pi c \xi) - \cos(2\pi b \xi)]
\end{split}\end{align} \par
\begin{align}\begin{split}
	\int_{b}^{c}t \sin(2\pi t \xi)dt &= \frac{1}{2\pi \xi} [t\cos(2\pi t \xi)|_b^c-\int_{b}^{c} \cos(2\pi t \xi)dt] \\
	&= \frac{1}{2\pi \xi}[c\cos(2\pi c \xi)-b\cos(2\pi b \xi)]+\frac{1}{(2\pi \xi)^2}[\sin(2\pi c \xi)-\sin(2\pi b \xi)]
\end{split}\end{align} \par
\begin{align}\begin{split}
	\int_{b}^{c}t^2 \sin(2\pi t \xi)dt &= \frac{1}{2\pi \xi} [t^2\cos(2\pi t \xi)|_b^c-2\int_{b}^{c} t\cos(2\pi t \xi)dt] \\
\end{split}\end{align} \par

\begin{align}\begin{split}
	\int_{a}^{b} e^{-2\pi i t \xi} dt = \frac{1}{-2\pi i \xi} e^{-2\pi i t \xi} |_a^b = \frac{1}{-2\pi i \xi}(e^{-2\pi i b \xi}-e^{-2\pi i a \xi})
\end{split}\end{align} \par
\begin{align}\begin{split}
	\int_{a}^{b} t e^{-2\pi i t \xi} dt &= \int_{a}^{b} \frac{1}{-2\pi i \xi} t \ de^{-2\pi i t \xi} \\
	&= \frac{1}{-2\pi i \xi} [te^{-2\pi i t \xi} |_a^b - \int_{a}^{b} e^{-2\pi i t \xi} dt ]\\
	&= \frac{1}{-2\pi i \xi} (be^{-2\pi i b \xi}-ae^{-2\pi i a \xi}+\frac{1}{2\pi i \xi}e^{-2\pi i b \xi}-\frac{1}{2\pi i \xi}e^{-2\pi i a \xi})
\end{split}\end{align} \par
\begin{align}\begin{split}
	\int_{a}^{b} t^2 e^{-2\pi i t \xi} dt &= \int_{a}^{b} \frac{1}{-2\pi i \xi} t^2 \ de^{-2\pi i t \xi} \\
	&= \frac{1}{-2\pi i \xi} [t^2 e^{-2\pi i t \xi} |_a^b - 2 \int_{a}^{b} te^{-2\pi i t \xi} dt ]\\
	&= \frac{1}{-2\pi i \xi}[(b^2e^{-2\pi i b \xi}-a^2e^{-2\pi i a \xi})\\
	&\ \ \ \ +\frac{1}{\pi i \xi}(be^{-2\pi i b \xi}-ae^{-2\pi i a \xi}+\frac{1}{2\pi i \xi}e^{-2\pi i b \xi}-\frac{1}{2\pi i \xi}e^{-2\pi i a \xi})]\\
	&= \frac{1}{-2\pi i \xi}[(b^2+\frac{1}{\pi i \xi}b+\frac{1}{2(\pi i \xi)^2})e^{-2\pi i b \xi}+(-a^2-\frac{1}{\pi i \xi}a-\frac{1}{2(\pi i \xi)^2})e^{-2\pi i a \xi}]
\end{split}\end{align} \par
\begin{align}\begin{split}
	\int_{a}^{b} t^3 e^{-2\pi i t \xi} dt &= \int_{a}^{b} \frac{1}{-2\pi i \xi} t^3 \ de^{-2\pi i t \xi} \\
	&= \frac{1}{-2\pi i \xi} [t^3 e^{-2\pi i t \xi} |_a^b - 3 \int_{a}^{b} t^2e^{-2\pi i t \xi} dt ]\\
	&= \frac{1}{-2\pi i \xi} \{b^3 e^{-2\pi i b \xi}-a^3 e^{-2\pi i a \xi} \\
	&\ \ \ \ + \frac{3}{2\pi i \xi}[(b^2+\frac{1}{\pi i \xi}b+\frac{1}{2(\pi i \xi)^2})e^{-2\pi i b \xi}+(-a^2-\frac{1}{\pi i \xi}a-\frac{1}{2(\pi i \xi)^2})e^{-2\pi i a \xi}]\}\\
	&= \frac{1}{-2\pi i \xi} [(b^3 + \frac{3}{2\pi i \xi}b^2+\frac{3}{2(\pi i \xi)^2}b+\frac{3}{4(\pi i \xi)^3})e^{-2\pi i b \xi}\\
	&\ \ \ \ - (a^3 + \frac{3}{2\pi i \xi}a^2+\frac{3}{2(\pi i \xi)^2}a+\frac{3}{4(\pi i \xi)^3})e^{-2\pi i a \xi}]
\end{split}\end{align} \par
%-----------------------------------------------
\subsection{Spatially Regularization $W$ \cite{danelljan2015learning}}
To resolve the problem of unwanted boundary effects of discriminatively learned correlation filters, a spatially regularization is introduced:
\begin{equation}
	w(m, n) = \mu + \eta(m/P)^2 + \eta(n/Q)^2
\end{equation}
where $P \times Q$ denotes the target size. \par
%-----------------------------------------------
\subsection{Tracking Frameworks}
\subsubsection{Localization}
\begin{enumerate}
	\item Extract the feature maps $\{x^d[n]\}: n \in \{0, \cdots, N_d -1\}, d \in \{1, \cdots, D\}$ from the region of interest in an image.
	\item Multiply it by the cosine window.
	\item Do DFT on the extracted feature maps:
		\begin{equation} 
			X^d[k] = \mathcal{F} (x^d[n])
		\end{equation} \par	
	\item Calculate $\widehat{S_f\{x\}}[k]$ according to (\ref{eq:scoreFourier}):
		\begin{equation} 
			\widehat{S_f\{x\}}[k]=\sum^D_{d=1} \hat{f}^d[k] X^d[k] \hat{b}_d[k]
		\end{equation} \par
	\item Calculate confidence score $s=S_f\{x\}$ by inverse DFT:
		\begin{equation} 
			s(t)=\mathcal{F}^{-1}(\widehat{S_f\{x\}}[k])
		\end{equation} \par
	\item To maximizing the score $s(t): t\in [0,T)$,
		\begin{enumerate}
			\item Using grid search on $s(T\frac{n}{2K+1})$ for $n=0, \cdots, 2K$ to find a rough
			 initial estimate $s(t)$.
			\item Do Fourier series expansion $s(t)=\sum^K_{-K}\hat{s}[k]e_k(t)$, and use the result above
			as the initial state, using Newtons method (\ref{ch:newtonmethod}) to do iterative optimization of it.
		\end{enumerate}
\end{enumerate}
\subsubsection{Training}
Using Conjugate Gradient (\ref{ch:conjugategradient}) to iteratively solving the equation (\ref{eq:normal}).
\subsubsection{Sampling method} \label{ch:samplingccot}
For each frame $j$, add one training sample $x_j$. The weights for each sample set to decay exponentially $\alpha_j \sim (1-\gamma)^{M-j}$. If the number of samples has reached a maximum number $M_{max}$, the sample with the smallest weight is replaced.
\subsubsection{Parameters}
Check file params.h in the code \url{https://github.com/rockkingjy/OpenTrackers}. 
%=======================================
\section{ECO \cite{DanelljanCVPR2017}} \label{ch:eco}
%-----------------------------------------------
\subsection{Factorized Convolution Operator}
\subsubsection{Factorized convolution}
$\hat{f}_i$ and $P_i$ with the under index $i$ means the value of the $i$th iteration. \par
Instead of learning one separate filter of each feature channel $d$, we use a smaller set of basis filters 
$f^1, \cdots, f^C$, where $C < D$. Then the filter for any feature layer $d$ could be constructed as: 
$f^d = \sum^C_{c=1} p_{d,c}f^c$. \par
Then the filter could be written as: $(Pf)_{D \times 1}$, so we have 
the factorized convolution operator from (\ref{eq:score}), with the linearity property of convolution, \par
\begin{equation}
	S_{Pf}\{x\}= Pf*J\{x\}=\sum^C_{c=1} \sum^D_{d=1} p_{d,c} f^c * J_d\{x^d\} = f*P^TJ\{x\}
\end{equation} 
where 
\begin{equation}
	f = 
	\begin{bmatrix}
		f^1 \\ \vdots \\ f^C
	\end{bmatrix}
	\text{, }
	P = 
	\begin{bmatrix}
		P_{1,1} & \cdots & P_{1,C} \\
		\vdots & \cdots & \vdots \\
		P_{D,1} & \cdots & P_{D,C} 
	\end{bmatrix}_{D \times C}
	\text{, }
	J\{x\} = 
	\begin{bmatrix}
		J\{x\}^1 \\ \vdots \\ J\{x\}^D
	\end{bmatrix}
\end{equation}
Here $P^T$ resembles a linear dimensionality reduction operator. \par
We now learning the filter $f$ and $P^T$ jointly. \par
Add a extra regularization using the Forbenius norm (ch. \ref{ch:forbenius}) of P, and now using just a \textbf{single} sample, the loss of (\ref{eq:target}) becomes: 
\begin{equation} \label{eq:targetFactorized}
	E(f, P)=\norm{S_{Pf}\{x\}-y}^2_{L^2}
		 + \sum^C_{c=1} \norm{w f^c}^2_{L^2}
		 + \lambda \norm{P}^2_F
\end{equation} \par
Do the Fourier transformation, 
\begin{equation}
	\widehat{S_{Pf}\{x\}} =\widehat{Pf*J\{x\}}
	=\sum^D_{d=1}\sum^C_{c=1}p_{d,c}\hat{f}^c[k]X^d[k]\hat{b}_d[k]
	=(\hat{z}^TP\hat{f})[k]
\end{equation}
where $\hat{z}^d[k]=\widehat{J_d\{x^d\}}[k]=X^d[k]\hat{b}_d[k]$. \par
By applying the Parseval's formula (\ref{eq:parseval}), 
\begin{equation} \label{eq:targetFourierFactorized}
	E(f, P)=\norm{\hat{z}^TP\hat{f}-\hat{y}}^2_{l^2}
		 + \sum^C_{c=1} \norm{\hat{w} * \hat{f}^c}^2_{l^2}
		 + \lambda \norm{P}^2_F
\end{equation} \par
We use Gauss-Newton (\ref{ch:gaussnewton})  and Conjugate Gradient (\ref{ch:conjugategradient}) methods to optimize the quadratic subproblems. The Gauss-Newton method is derived by linearizing the residuals in ({\ref{eq:targetFourierFactorized}) using a first order Taylor series expansion, 
\begin{align} \begin{split} \label{eq:approTayor}
	\hat{z}^T(P_i+\Delta P)(\hat{f}_i+\Delta\hat{f}) 
	&\approx \hat{z}^T P_i \hat{f}_{i, \Delta} +  \hat{z}^T \Delta P \hat{f}_{i} \\
	&= \hat{z}^T P_i \hat{f}_{i, \Delta} + (\hat{f}_i \otimes \hat{z})^T vec(\Delta P)
\end{split}\end{align} 
where $\hat{f}_{i, \Delta} = \hat{f}_i + \Delta \hat{f}$, and $\otimes$ is Kronecker product (ch. \ref{ch:kronecker}). \par
Substitute (\ref{eq:approTayor}) into (\ref{eq:targetFourierFactorized}):
\begin{equation} \label{eq:factorizedConvGaussNewton}
	\tilde{E}(\hat{f}_{i, \Delta}, P+\Delta P)
		=\norm{\hat{z}^TP_i\hat{f}_{i, \Delta}+ (\hat{f}_i \otimes \hat{z})^T vec(\Delta P)
 -\hat{y}_j}^2_{l^2}
		 + \sum^C_{c=1} \norm{\hat{w} * \hat{f}^c}^2_{l^2}
		 + \lambda \norm{P_i + \Delta P}^2_F
\end{equation} 
which is a linear least squares problem. \par
Then use Conjugate Gradient to optimize each Gauss-Newton subproblem to obtain the new filter 
$\hat{f}^*_{i, \Delta}$ and $\Delta P^*$, and the filter and matrix is updated as: $\hat{f}_{i+1}=\hat{f}^*_{i, \Delta}$
and $P_{i+1}=P_i+\Delta P^*$. \par
\subsubsection{Matrix version of Factorized Convolution Operator}
To make things clear, I present all the matrix here: 
\begin{equation}
	\hat{\bm{z}}^d= 
	\begin{bmatrix}
		\hat{z}^d[-K_d] \\  \vdots \\ \hat{z}^d[K_d] 
	\end{bmatrix}_{(2K_d+1) \times 1}
	\text{,   }
	\hat{\bm{z}}= 
	\begin{bmatrix}
		\hat{\bm{z}}^1 \\ \hat{\bm{z}}^2 \\ \vdots \\ \hat{\bm{z}}^D
	\end{bmatrix}_{\sum^{D}_{d=1}(2K_d+1) \times 1}
\end{equation} \par
\begin{equation}
	\hat{\bm{f}}_i^c= 
	\begin{bmatrix}
		\hat{f}^c_i[-K_c] \\  \vdots \\ \hat{f}^c_i[K_c] 
	\end{bmatrix}_{(2K_c+1) \times 1}
	\text{,   }
	\hat{\bm{f}}_i= 
	\begin{bmatrix}
		\hat{\bm{f}}^1_i \\ \hat{\bm{f}}^2_i \\ \vdots \\ \hat{\bm{f}}^C_i
	\end{bmatrix}_{\sum^{C}_{d=1}(2K_c+1) \times 1}
\end{equation} \par
\begin{equation}
	\hat{\bm{f}}_{i, \Delta}^c= 
	\begin{bmatrix}
		\hat{f}^c_{i, \Delta}[-K_c] \\  \vdots \\ \hat{f}^c_{i, \Delta}[K_c] 
	\end{bmatrix}_{(2K_c+1) \times 1}
	 \text{,   }
	\hat{\bm{f}}_{i, \Delta}= 
	\begin{bmatrix}
		\hat{\bm{f}}^1_{i, \Delta} \\ \hat{\bm{f}}^2_{i, \Delta} \\ \vdots \\ \hat{\bm{f}}^D_{i, \Delta}
	\end{bmatrix}_{\sum^{C}_{c=1}(2K_c+1) \times 1}	
\end{equation}
\begin{equation}
	 \hat{\bm{y}}_j= 
	\begin{bmatrix}
		\hat{y}_j[-K] \\  \vdots \\ \hat{y}_j[K] 
	\end{bmatrix}_{(2K+1) \times 1}
\end{equation}
\begin{equation}
	\bm{P}=
	\begin{bmatrix}
		p_{d, c}
	\end{bmatrix}_{\sum^{D}_{d=1}(2K_d+1) \times \sum^{C}_{c=1}(2K_c+1)}
	\text{, }
	\bm{P}\bm{\hat{f}_i}=
	\begin{bmatrix}
		p_{d, c}\bm{\hat{f}}^c_i 
	\end{bmatrix}_{\sum^{D}_{d=1}(2K_d+1) \times 1}
	\text{, }
	\bm{\hat{z}^TP\hat{f}_{i,\Delta}} \text{ is a scalar.}
\end{equation}\par
\begin{equation}
	\bm{\hat{f}}_i \otimes \bm{\hat{z}}=
	\begin{bmatrix}
		\bm{\hat{f}^c_i\hat{z}^d}
	\end{bmatrix}_{\sum^{C}_{c=1}(2K_c+1)\sum^{D}_{d=1}(2K_d+1) \times 1}
	\text{, }
	 B_f= 
	\begin{bmatrix}
		(\bm{\hat{f}}_i \otimes \bm{\hat{z}})[-K]^T \\  \vdots \\ (\bm{\hat{f}}_i \otimes \bm{\hat{z}})[K]^T
	\end{bmatrix}_{(2K+1) \times \sum^{C}_{c=1}(2K_c+1)\sum^{D}_{d=1}(2K_d+1)}
\end{equation}
where $K:=\max_dK_c$.  \par
\begin{equation}
 	\bm{p}= 
	\begin{bmatrix}
		vec(p_{d,c})
	\end{bmatrix}_{\sum^{C}_{c=1}(2K_c+1)\sum^{D}_{d=1}(2K_d+1) \times 1}
	 \text{,   }
	 \bm{\Delta p}= 
	\begin{bmatrix}
		vec(\Delta p_{d,c})
	\end{bmatrix}_{\sum^{C}_{c=1}(2K_c+1)\sum^{D}_{d=1}(2K_d+1) \times 1}
\end{equation}
\begin{equation}
	A^c_j = 
	\begin{bmatrix}
		X^c_j[-K_c]\hat{b}_c[-K_c] \bm{p}_c & \cdots & 0 \\
		\vdots & \ddots & \vdots \\
		0 & \cdots & X^c_j[K_c]\hat{b}_c[K_c]  \bm{p}_c\\
	\end{bmatrix}_{(2K_c+1)\times(2K_c+1)}
\end{equation} \par
\begin{equation}
	A_P = 
	\begin{bmatrix}
		0_{(K-K_1)\times(2K_1+1)} & \cdots & 0_{(K-K_C)\times(2K_C+1)}  \\
		A_j^1 & \cdots & A_j^C \\
		0_{(K-K_1)\times(2K_1+1)} & \cdots & 0_{(K-K_C)\times(2K_C+1)} 
	\end{bmatrix}_{(2K+1)\times \sum^C_{c=1}(2K_c+1)}
\end{equation} \par
The the equation (\ref{eq:factorizedConvGaussNewton}) becomes:
\begin{equation} \label{eq:targetDerivativeECO}
	\tilde{E}(\hat{\bm{f}}_{i, \Delta}, \bm{p + \Delta p})
		=\norm{A_P\hat{\bm{f}}_{i, \Delta}+ B_f \bm{\Delta p} -\hat{\bm{y}}}^2_2
		 + \norm{W \hat{\bm{f}}_{i, \Delta}}^2_2
		 + \lambda \norm{\bm{p} + \bm{\Delta p}}^2_2
\end{equation} \par
To expand it: 
\begin{align}\begin{split}
	\tilde{E}(\hat{\bm{f}}_{i, \Delta}, \bm{p + \Delta p}) 
		=&(A_P\hat{\bm{f}}_{i, \Delta}+ B_f \bm{\Delta p} -\hat{\bm{y}})^H(A_P\hat{\bm{f}}_{i, \Delta}+ B_f \bm{\Delta p} -\hat{\bm{y}}) \\
		& + (W \hat{\bm{f}}_{i, \Delta})^H(W \hat{\bm{f}}_{i, \Delta})
		 + \lambda (\bm{p} + \bm{\Delta p})^H(\bm{p} + \bm{\Delta p}) \\
		 =&(\hat{\bm{f}}_{i, \Delta}^HA_P^H+ \bm{\Delta p}^HB_f^H -\hat{\bm{y}}^H)(A_P\hat{\bm{f}}_{i, \Delta}+ B_f \bm{\Delta p} -\hat{\bm{y}}) \\
		& + (\hat{\bm{f}}_{i, \Delta}^HW^H)(W \hat{\bm{f}}_{i, \Delta})
		 + \lambda (\bm{p} ^H+ \bm{\Delta p}^H)(\bm{p} + \bm{\Delta p}) \\
		 =&(f^HAAf+f^HA^HB\Delta p - f^HA^Hy) + (\Delta p^HB^HAf + \Delta p^HB^HB\Delta p - \Delta p^HB^Hy) \\
		 &+(-y^HAf -y^HB\Delta p + y^Hy) + f^HW^HWf + \lambda (p^Hp+p^H\Delta p+\Delta p^H p + \Delta p^H\Delta p)
\end{split}\end{align} 
,the last formula removed the hats and index to make it clean. \par
Do the derivative to $\hat{\bm {f}}_{i, \Delta}$ and $\bm{\Delta p}$ for the equation by using the properties (\ref{eq:matrixderivativ1}) and (\ref{eq:matrixderivativ4}), and equals them to zero, then rearrange the equations, we have:
\begin{equation} \label{eq:eco}
	\begin{bmatrix}
		A^H_PA_P+W^HW & A^H_PB_f \\
		B^H_fA_P			& B^H_fB_f+\lambda I
	\end{bmatrix}
	\begin{bmatrix}
		\bm{\hat{f}_{i, \Delta}} \\
		\bm{\Delta p}
	\end{bmatrix}
	=
	\begin{bmatrix}
		A^H_P\bm{\hat{y}} \\
		B^H_f\bm{\hat{y}} -\lambda \bm{p}
	\end{bmatrix}
\end{equation} \par
Then we use Conjugate Gradient to iteratively solve this equation. 
%-----------------------------------------------
\subsection{Generative Sample Space Model}
The sampling method of C-COT is depicted in (ch. \ref{ch:samplingccot}), it has a issue of storing a large set of recent training samples. So a probabilistic generative model is proposed. \par
The approach is based on the joint probability distribution of $p(x,y)$. So, replacing (\ref{eq:target}) by:
\begin{equation} 
	E(f)=\mathbb{E} \{\norm{S_f\{x_j\}-y}^2_{L^2} \}+ \sum^D_{d=1} \norm{w f^d}^2_{L^2} 
\end{equation}
here $\mathbb{E}$ is evaluated over the distribution $p(x,y)$. If we suppose $p(x,y)=\sum^m_{j=1}\alpha_j\delta_{x_j,y_j}(x,y)$, then we get the original equation (\ref{eq:target}). \par
Notice that, the $y_j$ in (\ref{eq:target}) only differ by a translation, it equals the shifting of the feature map. Hence, we could assume that the target is centered in the image region and all $y=y_0$ are identical. Then the distribution can be factorized as $p(x,y)=p(x)\delta_{y_0}(y)$. Thus, we only need to estimate $p(x)$, employ Gaussian Mixture Model (GMM):
\begin{equation}
	p(x)=\sum^L_{l=1} \pi_l \mathcal{N} (x;\mu_l;I)
\end{equation}
where $L$ is the number of components, $\pi_l$ is the prior weight of component $l$, $\mu_l$ is its mean. $I$ is covariance matrix. \par
To update the GMM, if the number of components exceeds the $L$, we discard the component if its $\pi_l$ is below a threshold. Else, we merge the two closest components by:
\begin{equation} 
	\pi_n = \pi_k + \pi_l , \mu_n = \frac{\pi_k\mu_k+\pi_l\mu_l}{\pi_k+\pi_l}
\end{equation} \par
The distance of two components are calculated by $||\mu_k - \mu_l||$. Then the final loss is expressed as:
\begin{equation} 
	E(f)=\sum^L_{l=1} \pi_l \norm{S_f\{\mu_l\}-y_0}^2_{L^2} + \sum^D_{d=1} \norm{w f^d}^2_{L^2} 
\end{equation} \par
Compare to (\ref{eq:target}), it just do the following replacements:
\begin{equation} 
	m \rightarrow L \text{, } \alpha_j \rightarrow \pi_l \text{, } x_j \rightarrow \mu_l 
\end{equation}
so using the same method as C-COT to train it.
%-----------------------------------------------
\subsection{Update strategy}
Update the filter by starting the optimization process in every $N_s$th frame instead of every time.
%-----------------------------------------------
\subsection{Tracking Frameworks}
\subsubsection{Initialization for frame $0$ with bounding box}
\begin{enumerate}
	\item Initialize all the parameters.
	\item Extract features from the first frame (image):$\{x^d[n]: n \in \{0, \cdots, N_d -1\}, d \in \{1, \cdots, D\}\}$.
	\item Multiply the features by the cosine window.
	\item Do DFT on the features:
		\begin{equation} 
			X^d[k] = \mathcal{F} (x^d[n])
		\end{equation} \par
	\item Interpolate the features to the continuos domain:
		\begin{align}
			\widehat{J_d\{x^d\}}[k]&=X^d[k] \hat{b}_d[k] 
		\end{align} \par 
	\item Initialize projection matrix $P$, by using PCA to the interpolated features.
	\item Do the feature reduction for each feature channel $\hat{J}_d$:
	\begin{equation}
		\hat{J}_c = P^T\hat{J}_d
	\end{equation} 
	\item Initialize and update sample space.
	\item Calculate sample energy and projection map energy.
	\item Initialize filter $\hat{\bm{f}}_{0}$ and it's derivative $\Delta \hat{\bm{f}}_{0}$.
	\item Train the tracker(train the filter and update the projection matrix).
	\item Update the projection matrix $P$.
	\item Re-project the sample and update the sample space.
	\item Update distance matrix of sample space.
	\item Update filter  $\hat{\bm{f}}_{0}$.
\end{enumerate}
\subsubsection{Localization for frame $i$}
\begin{enumerate}
	\item Extract the features $\{x^d[n]: n \in \{0, \cdots, N_d -1\}, d \in \{1, \cdots, D\}\}$ from the region of interest in the frame $i$ for different scales.
	\item Do the feature reduction for each feature channel $x^d$:
	\begin{equation}
		 x^c = P^Tx^d
	\end{equation} 
	\item Multiply the feature by cosine window.
	\item Do DFT on the extracted features:
		\begin{equation} 
			X^c[k] = \mathcal{F} (x^c[n])
		\end{equation} \par	
	\item Interpolate the features to the continuos domain:
		\begin{align}
			\widehat{J_c\{x^c\}}[k]=X^c[k] \hat{b}_c[k] 
		\end{align} \par 
	\item Calculate score in Fourier domain $\widehat{S_f\{x\}}[k]$ according to (\ref{eq:scoreFourier}):
		\begin{equation}
			\widehat{S_f\{x\}}[k]=\sum^C_{c=1} \hat{f}^c[k] \widehat{J_c\{x^c\}}[k]
		\end{equation} \par
	\item Calculate confidence score $s=S_f\{x\}$ by inverse DFT:
		\begin{equation} 
			s(t)=\mathcal{F}^{-1}(\widehat{S_f\{x\}}[k])
		\end{equation} \par
	\item To maximizing the score $s(t): t\in [0,T)$,
		\begin{enumerate}
			\item Using grid search on $s(T\frac{n}{2K+1})$ for $n=0, \cdots, 2K$ to find a rough
			 initial estimate $s(t)$.
			\item Do Fourier series expansion $s(t)=\sum^K_{-K}\hat{s}[k]e_k(t)$, and use the result above
			as the initial state, using Newtons method (\ref{ch:newtonmethod}) to do iterative optimization of it.
		\end{enumerate}
	\item Update position and scale.
\end{enumerate}

\subsubsection{Training for frame $i$}
\begin{enumerate}
	\item Get the sample calculated in localization.
	\item Shift the sample so that the target is centered.
	\item Update sample space.
	\item Train the tracker every $N_s$th frame.
	\item Update the projection matrix $P$.
	\item Update filter  $\hat{\bm{f}}_{i}$.
\end{enumerate}
%=======================================
\section{Code analysis}
\subsection{parameters}
The key idea of tracker is to extract the features out from the target, so we have to define the feature first:
\begin{lstlisting}[language=C++]
	typedef std::vector<std::vector<cv::Mat>> ECO_FEATS;
	typedef cv::Vec<float, 2> COMPLEX;  
\end{lstlisting} \par
For each target given, we can extract several different features (ex. Deep feature, HOG feature, CN feature ect.), and for each feature, it may have several dimensions(ex. HOG normally is 31, CN is 11), so \textbf{ECO\_FEATS} the feature's type, with the structure of: [\#Features][\#Dimensions][Feature map], here \# means 'the number of'. \par
ECO is calculated in the domain of Complex number, so needs to define the type \textbf{COMPLEX} in c++. \par
%-----------------------------------------------
\subsection{sample\_update}
This file is for sample updating, and output two important values:
\begin{lstlisting}[language=C++]
	std::vector<ECO_FEATS> samples_f_;
	std::vector<float> prior_weights_;
\end{lstlisting} \par
\textbf{samples\_f\_} gives all the samples used for training, and \textbf{prior\_weights\_} gives the corresponding weights for the samples. \par
%-----------------------------------------------
\subsection{eco}
\textbf{yf\_} is the presumed target y in the Fourier domain. 
%-----------------------------------------------
\subsection{training}
\subsubsection{lhs\_operation}
This function is to calculate the left-hand side of the equation (\ref{eq:ccot}) (without update of projection matrix): \par
\begin{equation} 
	(A^H \Gamma A + W^H W) \hat{\bm{f}} = A^H \Gamma \hat{\bm{y}}
\end{equation} \par


%=======================================
\section{Some tips of transfer from matlab to $c++$}
\begin{itemize}
	\item Debug tricks, to show the line number, file number and functions:
		\begin{lstlisting}[language=C++]
#define debug(a, args...) printf("%s(%s:%d) " a "\n", 
	__func__, __FILE__, __LINE__, ##args)
#define ddebug(a, args...) printf("%s(%s:%d) " a "\n", 
	__func__, __FILE__, __LINE__, ##args)
		\end{lstlisting}
	\item Debug tricks, to show the property of cv::mat:
		\begin{lstlisting}[language=C++]
void imgInfo(cv::Mat mat)
{
	int type = mat.type();
	string r;

	uchar depth = type & CV_MAT_DEPTH_MASK;
	uchar chans = 1 + (type >> CV_CN_SHIFT);

	switch (depth)
	{
	case CV_8U:
		r = "8U";
		break;
	case CV_8S:
		r = "8S";
		break;
	case CV_16U:
		r = "16U";
		break;
	case CV_16S:
		r = "16S";
		break;
	case CV_32S:
		r = "32S";
		break;
	case CV_32F:
		r = "32F";
		break;
	case CV_64F:
		r = "64F";
		break;
	default:
		r = "User";
		break;
	}

	r += "C";
	r += (chans + '0');

	debug("%s %d x %d", r.c_str(), mat.rows, mat.cols);
	//return r;
}
		\end{lstlisting}		
	\item Std functions better to name out to prevent default link to other functions. 
	For example: 
		\begin{lstlisting}[language=C++]
void absTest()
{
	std::vector<float> v{0.1, 0.2};
	float abs = abs(1.23f);
	debug("False result:%f", abs);

	abs = std::abs(1.23f);
	debug("True result:%f", abs);
}
		\end{lstlisting}
	For the False result, it sometimes link to other library and give answer 1.
	\item Take care of the parameter types, better to write it out clearly.
		\begin{lstlisting}[language=C++]
void accumulateTest()
{
	std::vector<float> v{0.1, 0.2};
	float sum = std::accumulate(v.begin(), v.end(), 0);
	debug("False result:%f", sum);

	sum = std::accumulate(v.begin(), v.end(), 0.0f);
	debug("True result:%f", sum);
}
		\end{lstlisting}
	For the False result, it gave answer 0, while the True result is 0.3.
	\item In opencv, color order: BGR, in matlab: RGB.
	\item In opencv mat, the data store in this order: $channel -> x -> y$. Test example:
		\begin{lstlisting}[language=C++]
void opencvTest()
{
	float *newdata = (float *)malloc(sizeof(float) * (2 * 3 * 4));
	for (int i = 0; i < 2 * 3 * 4; i++)
	{
		newdata[i] = i;
	}

	cv::Mat mat = cv::Mat(cv::Size(3, 4), CV_32FC(2), newdata);

	printf("\nInfo of original mat:");
	imgInfo(mat);
	for (int i = 0; i < 2 * 3 * 4; i++)
	{
		printf("%f ", mat.at<float>(0, i));
	}
	printf("\n");

	std::vector<cv::Mat> splitmat;
	cv::split(mat, splitmat);

	printf("\nInfo of splited mat:");
	imgInfo(splitmat[0]);

	printf("channel 0:\n");
	for (int j = 0; j < mat.rows; j++)
	{
		for (int i = 0; i < mat.cols; i++)
		{
			printf("%f ", splitmat[0].at<float>(j, i));
		}
		printf("\n");
	}
	printf("\n");
	printf("channel 1:\n");
	for (int j = 0; j < mat.rows; j++)
	{
		for (int i = 0; i < mat.cols; i++)
		{
			printf("%f ", splitmat[1].at<float>(j, i));
		}
		printf("\n");
	}
}
		\end{lstlisting}
	\item When using libcaffe.so, using the BLVC caffe version. I used the Nvidia version and GPU 
	not working, wasted my days.
\end{itemize}
%=======================================
\renewcommand\refname{Reference}
\bibliographystyle{ieeetr} %unsrtnat
\bibliography{OpenTrackerNotes} 
\clearpage
\end{document}
